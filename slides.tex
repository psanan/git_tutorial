\documentclass{beamer}

%% TODO:

%I'm planning on covering
%- what git is
%- why you would want to use it
%- why it might be better than other version control systems you've used before
%- some of the annoying things to watch out for
%- How to use tools like github or bitbucket to be more productive
%
%I'll probably assume that you know how to use a terminal on your
%computer and can install the git executable (on linux, it's already
%there for you, on OS X you can get it from macports or homebrew, and
%on Windows I have no idea!)
%
%If you have any more specific requests about things you're interested
%in related to git (or version control in general), let me know before
%Wednesday.
%It'll be both.
%
%For those who aren't familiar, git is a "version control" system. The
%basic functionality is being able to save and record snapshots of your
%progress with a project, so that you can always return to a working
%state. Additional features include the ability to manage different
%versions of your project ("branches") and to work on the same project
%with other people using other computers.
%
%Such a system is extremely helpful for software projects, and is
%practically essential for software projects with more than one or two
%authors. It can also be used for any set of not-too-big files: a key
%example is working on a paper in latex.
%
%Github and Bitbucket are web-based services which keep a copy of your
%project (and its history), managed with git, on their servers and give
%you lots of nice tools to interact with it.
%
%If you find yourself wasting time doing these sorts of things, git can help:
%
% - Not being able to remember which is the original version of a piece of code
%
% - Accidentally overwriting your files
%
% - Getting confused sending code around via email
%
% - Ending up with two different versions of your code and not knowing
%how to merge them
%
% - Not being able to return to a working state of your code
%
% - Not being able to quickly work with remote collaborators on code
%
% - Being slowed down by a version control system which requires
%network access at all times




% Some misc notes:
%git:

%Where it comes from (parable) 
%eWhat you can do with it
%eWhy it’s better than subversion
%e  cheap branching
%eWhat’s bad about it, and how to mitigate those factors
%e  dumb syntax
%eGUIs and web tools
%eHow to learn more
%e
%eGood habits it encourages
%e  Working on atomic units of progress
%e  Summarizing and explaining what you’re doing
%e  Making your progress visible to other people [esp. remote collaborators]
%e  Writing to be shared and extended
%e  Moving between working states
%e  Backing up remotely
%e  Maintaining a canonical version (“upstream”) of a project
%e
%eThe main objective:
%e  Overcome the energy barrier to learn and become comfortable with the tools.




%\usepackage[backend=bibtex,maxnames=100]{biblatex} %from macports texlive-bibtex-extra
%\addbibresource{PFKrylov.bib}
%\renewcommand{\footnotesize}{\tiny}


\usetheme{Madrid}
\setbeamertemplate{navigation symbols}{} % Remove the navigation symbols
\setbeamertemplate{sections/subsections in toc}[default] %Turn off ugly toc numbering
\setbeamertemplate{itemize subitem}[triangle]
\setbeamertemplate{itemize item}[triangle]
\setbeamertemplate{enumerate items}[default]
%\setbeamercovered{transparent}
\usepackage{appendixnumberbeamer}
\defbeamertemplate{section page}{customsection}[1][]{%
  \begin{centering}
    {\usebeamerfont{section name}\usebeamercolor[fg]{section name}#1}
    \vskip1em\par
    \begin{beamercolorbox}[sep=12pt,center]{part title}
      \usebeamerfont{section title}\insertsection\par
    \end{beamercolorbox}
  \end{centering}
}
\defbeamertemplate{subsection page}{customsubsection}[1][]{%
  \begin{centering}
    {\usebeamerfont{subsection name}\usebeamercolor[fg]{subsection name}#1}
    \vskip1em\par
    \begin{beamercolorbox}[sep=8pt,center,#1]{part title}
      \usebeamerfont{subsection title}\insertsubsection\par
    \end{beamercolorbox}
  \end{centering}
}
\AtBeginSection{\frame{\sectionpage}}
%\AtBeginSubsection{\frame{\subsectionpage}}

%%%%%%%%%%%%%%%%%%%%%%%%%%%%%%%%%%%%%%%%%%%%%%%%%%%%%%%%%%%%%%%%%%%%%%%%%%%%%%%

\graphicspath{images/}

% clashes with beamer
%\usepackage[colorlinks=true]{hyperref}

\usepackage{booktabs}

%%%%%%%%%%%%%%%%%%%%%%%%%%%%%%%%%%%%%%%%%%%%%%%%%%%%%%%%%%%%%%%%%%%%%%%%%%%%%%%

%%%%%%%%%%%%%%%%%%%%%%%%%%%%%%%%%%%%%%%%%%%%%%%%%%%%%%%%%%%%%%%%%%%%%%%%%%%%%%%

\author{Patrick Sanan}
\institute[USI Lugano ICS / ETH Z\"{u}rich ERDW] 
{
USI Lugano / ETH Z\"{u}rich\\
}

%%%%%%%%%%%%%%%%%%%%%%%%%%%%%%%%%%%%%%%%%%%%%%%%%%%%%%%%%%%%%%%%%%%%%%%%%%%%%%%

\title{A Git Tutorial} 
\subtitle[]{What it is, how you use it, and what it's good for}
\date[]{} 
% Long date messes up footer

%%%%%%%%%%%%%%%%%%%%%%%%%%%%%%%%%%%%%%%%%%%%%%%%%%%%%%%%%%%%%%%%%%%%%%%%%%%%%%%

\begin{document}

%%%%%%%%%%%%%%%%%%%%%%%%%%%%%%%%%%%%%%%%%%%%%%%%%%%%%%%%%%%%%%%%%%%%%%%%%%%%%%%%
%%%%%%%%%%%%%%%%%%%%%%%%%%%%%%%%%%%%%%%%%%%%%%%%%%%%%%%%%%%%%%%%%%%%%%%%%%%%%%%%

\begin{frame}[fragile]
\titlepage 
\end{frame}

%%%%%%%%%%%%%%%%%%%%%%%%%%%%%%%%%%%%%%%%%%%%%%%%%%%%%%%%%%%%%%%%%%%%%%%%%%%%%%%

\begin{frame}
\tableofcontents 
\end{frame}

\section{Who am I?}
%% Can start with who the audience is. Specifically, it's the GFD group,
%%  but more generally it's the class of people who use computers in their work
%%  but are much more interested in what computers can do for them than how
%%  computers work. They can use a terminal, and are scientific, but 
%%  crucially they are NOT interested in cleverness, or hacking, or doing
%%  something because it's cool.

\section{What is git?}
%% Give what I just said, it might seem silly that I am going so start with 
%% a low-level description of what git does. Why? As much as I hate to do it, 
%% here's the xkcd cartoon.

%% I hated this cartoon when I saw it. Because I love git and think it makes total
%% sense. But I didn't always. So, let's start with ..
%%
%% A false history of version control [based loosely on ``a git parable'' ..
%% ...
%% [Along the way, we introduce functinality, which we label as {A},{B}, etc. Later, we show how this is practically manifested in git]
%% {A} Save labelled versions of your work, so that you can return to it at any time

%% {B} Maintain a 'tree' (better thought of as a DAG)

%% Now we should have a good idea of what git is

\section{How do I use it?}
%% So how do we use it????

%% 0. Getting it. This is often the hardest part about learning software. Getting it running on your machine.
%% 
%% I am going to assume now that you know how to use a terminal to navigate around your local file system [!! Am I going to assume this?? Hmm.. Depends how hands-on I want this to be ..]
%%

%% Due diligence for the examples
%% All examples are located at ...
%% The version of git used was XXX, on this operating system YYY..

%% 1. Working on your own local project
%%
%% [Here and elsewhere, we refer back to the abstract things that we set up in the first section]
%% {A}, saving labelled version of your own work.
%% ...
%% {B} As you work, you build up a tree (a linear one for now) 
%%
%% How can you see this tree?
%%%

%% ...

%% Remote tools ..

%% So now that I can use this thing, how will it make my life better? 

%% Wiping out a file completely

\section{What can I do with it?}

%% Working with latex. Here's a .gitignore. Recommendation: put sentences on new lines.

%% Working on code which needs to run on many different machines. 

%% Having a canonical version of something.

%% Free backups and transfers!

%% Coordinating with a team

%% Contributing to a project



\end{document}
