\documentclass{beamer}
\usepackage{algorithm}
\usepackage[noend]{algpseudocode}
\usepackage{listings}
\usepackage{courier}
\lstset{
  language=C++,
  basicstyle=\footnotesize\ttfamily,
  numberstyle=\tiny,
  numbersep=5pt,
  tabsize=2,
  extendedchars=true,
  breaklines=true,
  %frame=b,
  keywordstyle=\color{red},
  stringstyle=\color{blue}\ttfamily,
  commentstyle=\color{greenmod},
  numberstyle=\color{violet},
  identifierstyle=\color{orange},
  showspaces=false,
  showtabs=false,
  xleftmargin=17pt,
  framexleftmargin=17pt,
  framexrightmargin=5pt,
  framexbottommargin=4pt,
  showstringspaces=false
}

%% Remember: keep this PRACTICAL. How do you literally do all these things?

%% TODO:

%I'm planning on covering
%- what git is
%- why you would want to use it
%- why it might be better than other version control systems you've used before
%- some of the annoying things to watch out for
%- How to use tools like github or bitbucket to be more productive
%
%I'll probably assume that you know how to use a terminal on your
%computer and can install the git executable (on linux, it's already
%there for you, on OS X you can get it from macports or homebrew, and
%on Windows I have no idea!)
%
%If you have any more specific requests about things you're interested
%in related to git (or version control in general), let me know before
%Wednesday.
%It'll be both.
%
%For those who aren't familiar, git is a "version control" system. The
%basic functionality is being able to save and record snapshots of your
%progress with a project, so that you can always return to a working
%state. Additional features include the ability to manage different
%versions of your project ("branches") and to work on the same project
%with other people using other computers.
%
%Such a system is extremely helpful for software projects, and is
%practically essential for software projects with more than one or two
%authors. It can also be used for any set of not-too-big files: a key
%example is working on a paper in latex.
%
%Github and Bitbucket are web-based services which keep a copy of your
%project (and its history), managed with git, on their servers and give
%you lots of nice tools to interact with it.
%
%If you find yourself wasting time doing these sorts of things, git can help:
%
% - Not being able to remember which is the original version of a piece of code
%
% - Accidentally overwriting your files
%
% - Getting confused sending code around via email
%
% - Ending up with two different versions of your code and not knowing
%how to merge them
%
% - Not being able to return to a working state of your code
%
% - Not being able to quickly work with remote collaborators on code
%
% - Being slowed down by a version control system which requires
%network access at all times




% Some misc notes:
%git:

%Where it comes from (parable) 
%eWhat you can do with it
%eWhy it’s better than subversion
%e  cheap branching
%eWhat’s bad about it, and how to mitigate those factors
%e  dumb syntax
%eGUIs and web tools
%eHow to learn more
%e
%eGood habits it encourages
%e  Working on atomic units of progress
%e  Summarizing and explaining what you’re doing
%e  Making your progress visible to other people [esp. remote collaborators]
%e  Writing to be shared and extended
%e  Moving between working states
%e  Backing up remotely
%e  Maintaining a canonical version (“upstream”) of a project
%e
%eThe main objective:
%e  Overcome the energy barrier to learn and become comfortable with the tools.




%\usepackage[backend=bibtex,maxnames=100]{biblatex} %from macports texlive-bibtex-extra
%\addbibresource{PFKrylov.bib}
%\renewcommand{\footnotesize}{\tiny}


\usetheme{Madrid}
\setbeamertemplate{navigation symbols}{} % Remove the navigation symbols
\setbeamertemplate{sections/subsections in toc}[default] %Turn off ugly toc numbering
\setbeamertemplate{itemize subitem}[triangle]
\setbeamertemplate{itemize item}[triangle]
\setbeamertemplate{enumerate items}[default]
%\setbeamercovered{transparent}
\usepackage{appendixnumberbeamer}
\defbeamertemplate{section page}{customsection}[1][]{%
  \begin{centering}
    {\usebeamerfont{section name}\usebeamercolor[fg]{section name}#1}
    \vskip1em\par
    \begin{beamercolorbox}[sep=12pt,center]{part title}
      \usebeamerfont{section title}\insertsection\par
    \end{beamercolorbox}
  \end{centering}
}
\defbeamertemplate{subsection page}{customsubsection}[1][]{%
  \begin{centering}
    {\usebeamerfont{subsection name}\usebeamercolor[fg]{subsection name}#1}
    \vskip1em\par
    \begin{beamercolorbox}[sep=8pt,center,#1]{part title}
      \usebeamerfont{subsection title}\insertsubsection\par
    \end{beamercolorbox}
  \end{centering}
}
\AtBeginSection{\frame{\sectionpage}}
%\AtBeginSubsection{\frame{\subsectionpage}}

%%%%%%%%%%%%%%%%%%%%%%%%%%%%%%%%%%%%%%%%%%%%%%%%%%%%%%%%%%%%%%%%%%%%%%%%%%%%%%%

\graphicspath{images/}

% clashes with beamer
%\usepackage[colorlinks=true]{hyperref}

\usepackage{booktabs}

%%%%%%%%%%%%%%%%%%%%%%%%%%%%%%%%%%%%%%%%%%%%%%%%%%%%%%%%%%%%%%%%%%%%%%%%%%%%%%%

%%%%%%%%%%%%%%%%%%%%%%%%%%%%%%%%%%%%%%%%%%%%%%%%%%%%%%%%%%%%%%%%%%%%%%%%%%%%%%%

\author{Patrick Sanan}
\institute[USI Lugano ICS / ETH Z\"{u}rich ERDW] 
{
USI Lugano / ETH Z\"{u}rich\\
}

%%%%%%%%%%%%%%%%%%%%%%%%%%%%%%%%%%%%%%%%%%%%%%%%%%%%%%%%%%%%%%%%%%%%%%%%%%%%%%%

\title{A Git Tutorial} 
\subtitle[]{What it is, how you use it, and what it's good for}
\date[]{} 
% Long date messes up footer

%%%%%%%%%%%%%%%%%%%%%%%%%%%%%%%%%%%%%%%%%%%%%%%%%%%%%%%%%%%%%%%%%%%%%%%%%%%%%%%

\begin{document}

%%%%%%%%%%%%%%%%%%%%%%%%%%%%%%%%%%%%%%%%%%%%%%%%%%%%%%%%%%%%%%%%%%%%%%%%%%%%%%%%
%%%%%%%%%%%%%%%%%%%%%%%%%%%%%%%%%%%%%%%%%%%%%%%%%%%%%%%%%%%%%%%%%%%%%%%%%%%%%%%%

\begin{frame}[fragile]
\titlepage 
\end{frame}

%%%%%%%%%%%%%%%%%%%%%%%%%%%%%%%%%%%%%%%%%%%%%%%%%%%%%%%%%%%%%%%%%%%%%%%%%%%%%%%

\begin{frame}
\tableofcontents 
\end{frame}

\section{Who is this for?}
\begin{frame}[fragile]
\frametitle{}
Assumptions about the audience:
\begin{itemize}
\item You use code
\item You sometimes feel like you're wasting your time when you work with code, especially when it comes time to collaborate
\item You know how to use a terminal and login file on your computer \footnote{If not, ask me afterwards}
\item You don't care much about the inner workings of git, as long as it functions
\end{itemize}
\end{frame}


% As we proceed, I will use a two examples:
%1. a toy code that we've been working with.
% 2. an extremely simple project that we'll start from scratch

% You can find all this material (tracked with git of course), at
%  bitbucket.org/psanan/gittutorial
% ?? Actually, move this to github..


\section{What is git?}
\begin{frame}[fragile]
\frametitle{Why Version Control?}
git is a \emph{version control system}.

We will embark on a fictional history of git, a story of how it might have been invented, based on ``a git parable'' by Tom Preston-Warner.
\end{frame}

\begin{frame}[fragile]
\frametitle{Tracking History}
% Picture here: series of folders
% "Snapshots"
\end{frame}

\begin{frame}[fragile]
\frametitle{Adding Metadata}
% Picture : file in each folder with info
% "Commits"

% A simple extension is to have a merge commit, which has two parents
\end{frame}

\begin{frame}[fragile]
\frametitle{Different Versions}
% With the commit data above, our history no longer has to be linear:
% "branches"
% We can keep track of different branches which point to a commit
% Picture with branches file
\end{frame}

\begin{frame}[fragile]
\frametitle{Other People}
 Now suppose I want to work on the same code as another person
% 
% "repositories" and "remote repositories"
I define what sort of data is in a \emph{repository}, and how its stored. The key property is that the exact same changes to the data result in identical storage in the repository. 

I also define a \emph{remote repository} which is a reference, in my repository, to another one.

Finally, I introduce tools to merge two branches together.

Now, I can send a branch from my repository to another one, where they can merge it with one of their branches.

I can create a folder which I use to hold your branches. I then merge them with my branches, creating a new commit, and you then in turn fetch my version and merge it with your version. Now, we both have the same version!

% Picture : Two folders, my branches and your remote branches and data. Note that we agreed on the file naming, so I only need to copy over snapshots that I don't have!

\end{frame}

\begin{frame}[fragile]
\frametitle{That's it!}
Git is simply software that, at its core, does all the things we just mentioned, properly implemented:
\begin{itemize}
\item Defines data in a \emph{repository}
\item Records \emph{snapshots} of a set of files
\item Gives you a way to obtain a copy of the files, edit them, and add your changes as a new snapshot.
\item Keeps track of \emph{commits} for each snapshot: a summray, who made the changes, when, and from what previous snapshot.
\item Keeps track of pointers to commits called \emph{branches} and a way to increment them as new commits are added, and a way to merge them.
\item Keeps track of references to \emph{remote repositories} and provides a way to send and recieve branches to them.
\end{itemize}
\end{frame}

\begin{frame}[fragile]
\frametitle{Aside: Differences from other Version Control Systems}
Some of you may be scarred by previous version control systems.
\begin{itemize}
\item It's very easy to set up a repostory and share it.
\item Git (and Mercurial/Hg) are \emph{distributed version control systems}. There is no need to be in constant contact with a central server, you have a copy of everything on your machine, and branching is cheap
\item It's hard to lose data with git. You have a complete copy of the history locally. The files in the \texttt{.git} folder are all named with cryptographic hashes, so if they are corrupted, you will find out.
\end{itemize}
\end{frame}

\begin{frame}[fragile]
\frametitle{A Git Repository in Graphical Form}
% Picture: a tree with two merging branches, master and psanan/feature-update.
% Have snapshots as things external to the commits, which form the tree, and make sure
% it's obvious that HEAD and branch names are just pointers into the tree

\end{frame}


\section{How do I use it?}
\begin{frame}[fragile]
\frametitle{How do I use it?}

\end{frame}

\begin{frame}[fragile]
\frametitle{Obtaining Git}
There are many ways to obtain git. 
You can download a binary from \texttt{git-scm.com}, if you are using Ubuntu, you likely already have it, and if you used MacPorts or Homebrew on OS X you can install it there.
\end{frame}

\begin{frame}[fragile]
\frametitle{Setting Git Up}
To be able to keep track of who made which changes, git needs to know who you are.
%% Setting it up. You need to tell git who you are. [crib here from Lecture 1 and from git book]
\end{frame}

\begin{frame}[fragile]
\frametitle{Creating a Repository}
From now, we'll start working on a real example. I assume that I have a working \lstinline{git} executable and that I have set up my login file to establish my identity.
I create a new directory and create a new git repository there:
\begin{lstlisting}[language=C++]
mkdir myDemoProject
cd myDemoProject
git init
ls .git
\end{lstlisting}
\end{frame}
(Note: I'll be doing everything from the terminal here, but you can also use various GUI tools to work with git: \texttt{git-scm.com/downloads/guis})

\begin{frame}[fragile]
\frametitle{Adding and Tracking changes to files}
\begin{lstlisting}[language=C++]
vim data.txt
git status
git add data.txt
git status
git commit
git log
\end{lstlisting}
\end{frame}

\begin{frame}[fragile]
\frametitle{Writing Good Commit Messages}
\begin{verbatim}
Component: summary

After a blank line, describe what you did. 
This will be something read later on by you,
and by other people trying to figure out what
broke their code. If you did something that could 
cause problems for someone, note it here. It's
also a good idea to wrap the lines yourself.
\end{verbatim}
\end{frame}

\begin{frame}[fragile]
\frametitle{What's in a commit?}
\begin{itemize}
\item For basic usage, anything you changed since the last time your code worked.
\item For work in teams or on larger projects,
\begin{itemize}
\item Changes related to a particular task on a particular component
\item Something that is reasonably atomic
\item Something which won't interfere with other people's work without cause
\end{itemize}
\end{itemize}
\end{frame}

\begin{frame}[fragile]
\frametitle{Branching}
By default you are on a branch called \texttt{master}
\begin{lstlisting}[language=C++]
git branch psanan/data-reorganize
git checkout psanan/data-reorganize
vim data.txt
git add data.txt
git commit
git log
git checkout master
vim data.txt
git log
\end{lstlisting}
A good way to name branches is \lstinline{yourname/component-description}.
\end{frame}


% Make sure this text example is also with the files for this presentation
%% Due diligence for the examples
%% All examples are located at ...
%% The version of git used was XXX, on this operating system YYY..


% The three areas
%% While working, it's useful to think of each file (or part of a file) as being in one of three states
%% Committed, Modified, and Staged
% We can add a fourth state as untracked

% HEAD?? 
% You will see this a lot. It's just a pointer to a branch, and a branch is just a pointer to a commit!
% This points to ``where you are'', what snapshot you are working on in your working directory.
\begin{lstlisting}[language=C++]
cat .git/HEAD
cat .git/refs/heads/master
\end{lstlisting}

% Making branches
%% make them often!
%% Name them like yourname/feature-description
%%% Important to know who is in charge of a branch, and what it affects

%% Command cheat sheet.
%%% One of the most common complaints about git is that the command are irritating to remember. With that in mind, here is a small set worth memorizing

% git help
% git help [command]

% git init . This is so simple that it's worth tracking even the smallest project (track your resume, track your login files, track your homework assignments, ..)

% git status . Your best friend. This allows me to ignore many of the other things I might write on this list!

% git diff [filename] . What have you changed?

% git log -10

% get fetch . An extremely common confusin is related to the fact that you have a LOCAL copy of REMOTE branches on your machine. These are NOT automatically synchronized!

% git pull

% git push

% git push -u remotename branchname

% git commit

% git branch

% git checkout

% git remote -vv


% Panic!! What to do if you get into trouble??
%% Relax. Most things in git are non-destructive. Exceptions: thing you have to ``force'', rebasing, hard resetting.
%% git status will often tell you how to abort something (say, a merge).

%% Remote tools ..



\begin{frame}[fragile]
\frametitle{For More}
The Git Book : \href{https://git-scm.com/book}{git-scm.com/book}
\end{frame}

\subsection{Remote Tools}

% The basics: setting up a project on bitbucket and github, pushing to it, pulling from it, creating a branch there
% Pro Tip: SSH keys. This is really worth the hassle - typing in a password all the time is a constant disincentive to push.

% The two roles of history:
%% 1. A record of what's happened [never rewrite history!]
%% 2. A way to organize code development [rewrite history!]

% A rule of thumb to deal with this tension is the following
%% The typical workflow goes as such:
%% There is some 'upstream' version of the code (usually the one on github/bitbucket)
%% One developer wants to do some work
%% They:
%% 1. clone the repository, or if they already have it, check out the master branch and run \lstinline{git pull}.
%% 2. create a new branch \lstinline{git checkout -b psanan}
%% 3. Do some work. edit edit add commit edit edit add commit edit add commit
%% 4. Push the branch to the remote repository and track it \lstinline{git push -u origin myname/feature-description}

%% So now that I can use this thing, how will it make my life better? 

%% Wiping out a file completely

%%%%%%%%%%% DEMO 1 ** split this up!! **
% We start assuming that git is installed and that you have a login file
% 0. Set up your login file
% 1. Create a repo
% 2. Add some stuff
% 3. Create an account of github
% 4. Add an SSH key 
% 5. Push my project to github
% 6. Create a branch
% 7. Push it to github
% 8. Work on both branches
% 9. Merge my branches

\section{What can I do with it?}

%% Working on code which needs to run on many different machines. 

%% Having a canonical version of something.

%% Free backups and transfers!

%% Coordinating with a team

%% Contributing to a project

%% Organize your work

%%%%%%%%%%%% DEMO 2
% The idea here is how to join an existing projec
% Check out our geophysics code [do on github so I can learn that..]
%  create our own branch
% work on it
% push it
% Comment on it, tagging other users
% Merge it
% Look at the history of a file

\section{Annoyances and Solutions}
\begin{frame}[fragile]
\frametitle{How do I remember all these stupid commands?!}
\includegraphics[width=100px]{git.png}
http://xkcd.com/1597/
\end{frame}

% The syntax
%% People often complain (see git xkcd)

% It's a useful enough tool to justify learning: there are a few commands you will use all the time,
%  and the rest you can look up
% git status tells you a lot
% git help [function] tells you a lot
% make sure you understand the underlying models. Commits, the staging area, branches, remotes.

% A GUI tool is a useful tool, but it really helps to know the model underlying git
%  For things like commiting parts of files and reviewing your change sets, these tools are great.
% GitX, GitK, GitHub, SourceTree, git-gui ..
% Even MATLAB seems to have some git functionality these days

% Merges
%% Pull often
%% Make commits on contained things

% What's the state??
%% git status
%% git diff
%% git prompt (just give link)

%% Uncommited work when you want to pull
%% Work on contained commits. Git forces you to work in a more focused way.
%% You can always reset and amend your commits, so advanced users can just commit things and then undo them.
%% git stash: dangerous! (but very useful sometimes..)
This is the only ``dangerous'' thing I will discuss today.
\begin{lstlisting}[language=C++]
git stash
git stash apply
git stash clear
\end{lstlisting}





\end{document}
